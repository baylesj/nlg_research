\documentclass[final]{beamer} % use beamer
\usepackage[orientation=landscape,size=a0,scale=1.2]{beamerposter} % beamer in poster size
\usepackage{graphicx}
\usepackage{amsfonts}
\newcommand{\nn}{\nonumber}
\renewcommand{\phi}{\varphi}
%\newcommand{\thispdfpagelabel}{}
\newcommand{\comment}[1]{}
\newcommand{\field}[1]{\mathbb{#1}} % requires amsfonts
\newcommand{\pd}[2]{\frac{\partial#1}{\partial#2}}
\newcommand{\script}[1]{\mathcal{#1}} % requires amsfonts
\newcommand{\am}{\bar{\bar{\alpha}}}
\usetheme{posterosu}            % our poster style

\title{A polynomial chaos-based method for the continuous spectrum biphasic poroviscoelastic model of articular cartilage}
\author[*baylesj@onid.orst.edu]{Jordan Bayles*, Prof. Nathan Gibson}
\institute[Department of Mathematics, Oregon State University]{Department of Mathematics, Oregon State University}
%\date{IMA 2011 Workshop: Large-scale Inverse Problems and Quantification of Uncertainty}

\begin{document}

\begin{frame}[t]

\begin{columns}[t]
\begin{column}{0.28\paperwidth}

\begin{alertblock}{Abstract}
We present a numerical method for solving the linear biphasic poroviscoelastic
model of articular cartilage, specifically by 
%reducing the solution from the partial differential equation 
replacing the convolution representing history effects with an
%governing this model to a simpler, 
auxiliary ODE
to which generalized Polynomial Chaos (gPC) is applied to account for
a continuous spectrum of relaxation times.
The
resulting system of 
ODEs is solved using a finite difference method of the same order of
accuracy as the method for the PDE. The results of this solution
are then compared to existing literature to determine overall accuracy and
the previous, Gauss-Legendre solution, is recovered as a special case of gPC.

\end{alertblock}
This is a URISC funded undergraduate research project with an emphasis
on reformulation of the model and implementation of
gPC, and can be considered as an extension of the work done in \cite{stuebner},
\cite{schugart} as well as an application of new methods in electromagnetics. 
Note that this project is still in progress.

\begin{block}{Application}
The ultimate purpose of this project is to improve the simplicity, efficiency, and
utility of the BPVE model of articular cartilage. Upon doing so, this will make further
studies into the behaviors of cartilage better informed, and add value to not only
mathematical literature but medical as well, furthering joint research.
\end{block}

\begin{block}{Primary Problem}
We desire a solution $u$, to the linear biphasic viscoporoelastic (BPVE) model
of articular cartilage
\[
\frac{1}{\kappa} \pd{u}{t} = \pd{\sigma}{z},\; 0<z<h,\; 0<t<t_f.
\]
Where the stress, $\sigma$, is defined to be
\[
\sigma(z,t) = H_A \int_{-\infty}^{t} G(t-s)  \pd{\epsilon}{s} ds.
\]
The quantity $\epsilon(z,t) = \pd{u}{z}$ is the strain. Substituting
the definition of 
$\epsilon(z,t)$ into the original model and rewriting in terms of the
velocity $v=\pd{u}{t}$, then
\[
 v = \kappa H_A \int_{-\infty}^{t}G(t-s) \pd{^2 v}{z^2} (z, s) ds.
\]
where typically $G(t) = G_{\infty} + G_d \, g(t;\tau)$ and 
\[
g(t;\tau)=\frac{e^{-t/\tau}}{\tau}.
\]
This method was implemented by \cite{stuebner,schugart}, however we
wish to employ an auxiliary ordinary differential equation (ODE)
approach. To that end we define $Q(z,t) := G_d \, g(t; \tau) *
v_{zz}$. Taking a time derivative on both sides gives
\[
\pd{v}{t} = \kappa H_A G_{\infty} \pd{^2 v}{z^2} + \kappa H_A \dot{Q}.
\]
We further introduce a new variable $P$ such that $\pd{^2 P}{z^2} = Q$ and
\[
P(z, t) := G_d \, g(t; \tau) * v(z,t).
\]
Note that $P$ satisfies the auxiliary ODE
\[
\tau\dot{P} +P = G_d v.
\]
After substituting $\dot{Q}=\pd{^2 \dot{P}}{z^2}$, the original PDE becomes
\[
\pd{v}{t} = \kappa H_A \left(G_{\infty} + \frac{G_d}{\tau}\right) \pd{v}{z} - 
\frac{\kappa H_A}{\tau} \pd{^2 P}{z^2}
\]
\end{block}

\end{column}
\begin{column}{0.28\paperwidth}

\begin{alertblock}{Development of New Method}
The above assumes a single relaxation time parameter $\tau$. In the
presence of a continuous spectrum of relaxation times, one may redefine the kernel in
the convolution integral to be
\[
G(t) = G_{\infty} + G_d \, \int_{\tau_1}^{\tau_2} \frac{e^{-t/\tau}}{\tau} dF(\tau)
\]
where $F(\tau)$ is the probability density of relaxation times on the interval $[\tau_1,
\tau_2]$. In previous efforts, this kernel was approximated via Gaussian quadrature
in order to produce $G^Q (t)$ with arbitrary order of accuracy. Here we propose to use
the approach defined in \cite{gibson}, where we explicitly model the relaxation as a
random variable. Then polynomial chaos is applied to the resulting random solution.
\end{alertblock}

\begin{block}{Polynomial Chaos}
We use polynomial chaos expansions for the function $\script{P}$, defined analogously
to $P$ as $\script{P} (z, t) := G_d g(t; \tau) * v$ and satisfying the ODE
\begin{equation}
\tau \dot{\script{P}} + \script{P} = G_d v \;\rm{with}\; \tau \sim F.
\end{equation}
Applying gPC, we can rewrite as $\tau = r\xi+m$, where $\xi \in [-1,1]$.
and approximate the solution 
\begin{eqnarray}
\script{P}(t,z,\xi) & \approx & \sum_{j=0}^{p} \alpha_j (t,z) \phi_j(\xi)
\label{eqn2}
\end{eqnarray}
where the expansion coefficients satisfy the following tridiagonal
system of ODEs
\begin{equation} \label{eqn:alpha_m}
A \dot{\vec{\alpha}} + \vec{\alpha} = \vec{f}.
\end{equation} 
%\[(rM +mI) \dot{\vec{\alpha}} + \vec{\alpha} = G_d v(t,z) \hat{e}_1 \]
The PDE depends on the quantity $\field{E}_F \left[\script{P}\right] = \alpha_0 (t,z)$.
The $\phi_j$'s above are (for example) Jacobi polynomials in $\xi$,
which are orthogonal on $[-1,1]$ with respect to a weight function
proportional to the distribution $F$, e.g.,
\[W(\xi) = (1-\xi)^a (1+\xi)^b.\]
\end{block}

\begin{block}{System of Equations }
Equation $\ref{eqn:alpha_m}$ represents a system of equations expressed in matrix form,
where
\[
A := (rM + mI),\;
\]
\[
\vec{f} := \left[\begin{array}{c}
G_d v(t,z) \hat{e}_1\\
0\\
\vdots\\
0
\end{array}\right],\;
\]
\[
M := \left[\begin{array}{ccccc}
b_0 & a_1 & 0 & \dots & 0 \\
c_0 & b_1 & \ddots &  & \vdots \\
0 & c_1 & \ddots & a_{Q-1} & 0 \\
\vdots &  & \ddots & b_{p-1} & a_p \\
0 & \dots & 0 & c_{p-1} & b_p
\end{array}\right].
\]
and $a_i, b_i, c_i$ are the recursion coefficients. Note that the deterministic
value $\vec{f}$ forces the system and is dependent on $v(t,z)$, which is itself
dependent on the expected value $P = \field{E}\left[\script{P}\right]$, which can
be approximated by $\alpha_0$.
\end{block}

\begin{block}{Alternative to Numerical Integration}
\begin{itemize}
\item
Direct approach to approximating solutions to the ODEs for $\script{P}$ 
is to compute inner products,
$\left<\phi_i,\phi_j\right>_W$, with numerical integration (quadrature)
\item
Numerical integration needed at each time $t$ and can be expensive
\item
Instead, we use an efficient method for approximating $\alpha_j$ that 
avoids computing the integrals directly in favor of a finite difference approach.
\end{itemize}
\end{block}
\end{column}

\begin{column}{0.28\paperwidth}
%\begin{block}{Assumptions*}
%
%\begin{eqnarray}
%g(t,\xi) &=& \sum_{k=0}^{\infty} G_k(t) \xi^k.
%\label{eqn5}
%\end{eqnarray}
%Approximating modes in (\ref{eqn4}) begins with a truncation of the series in (\ref{eqn5})
%\begin{eqnarray}
%g^Q(t,\xi) := \sum_{k=0}^{Q} G_k(t) \xi^k.
%\label{eqn6}
%\end{eqnarray}
%*Taylor expansions and general polynomial bases can be handled by converting to the form $(\ref{eqn5}$) or
%by using nested polynomial representations.
%\end{block}
\begin{block}{Iterative method for determining $\alpha_j$}
Using matrix notation:
\[
\alpha_j(t_n, z_i) = [\bar{\bar{\alpha}}_{ij}^n] = 
\left[ \begin{array}{c c c c}
\alpha_0 (z_0) & \alpha_1^n(z_0) & \cdots & \alpha_p^n (z_0)\\
\alpha_0 (z_1) & \cdots          &        &                 \\
\vdots          &                &        &                 \\
\alpha_0 (z_M)  &                &        &                 \\
\end{array}\right] =
\bar{\bar{\alpha}}^n
\]

Using finite difference approximation, \ref{eqn:alpha_m} can be written as 
\begin{equation}
(rM + mI) \frac{(\am^{n+1})^T - (\am^n)^T}{\Delta t} + (\am^{n+\theta})^T =
G_d \hat{e}_1 (\vec{v}^{n+\theta})^T
\end{equation}
Let
\[ \begin{array}{c}
B_+ := rM + mI + \Delta t \theta I \\
B_- := rM + mI - \Delta t (1-\theta) I
\end{array} \]
Then this simply becomes
\begin{equation}
B_+ (\am^{n+1})^T = B_- (\am^n)^T + G_d \hat{e}_1 (\vec{v}^{n+\theta})^T \Delta t
\end{equation}
Solving for $\am^{n+1}$ this becomes
\begin{equation}
\am^{n+1} = \am^n B_-^T B_+^T + G_d \vec{v}^{n+\theta} \hat{e}_1^T B_+^{-T} \Delta t
\end{equation}
However, we are ultimately interested in $P$, not $\alpha$, thus we extract the first column from the $\alpha$ matrix.
\end{block}

\begin{block}{Current, Future Work}
\begin{itemize}
\item
Finish MATLAB implementation of BPVE model, gPC application numerical method.
\item
Compare results of numerical method to those utilized in \cite{stuebner}, \cite{schugart}.
\item
Determine error bounds in order to find the truncation level of the orthogonal
polynomials used in gPC in order to reach and surpass the accuracy of existing methods.
\item
Beginning with the uniform distribution, apply various probability densities in order
to determine the best fit for the relaxation time and numerical practicality.
\end{itemize}
\end{block}

\begin{block}{References}
\begin{thebibliography}{7}

{\small
\bibitem{stuebner}[2010] M. Stuebner and M. A. Haider, A fast quadrature-based numerical method for the continuous spectrum biphasic poroviscoelastic model of articular cartilage, {\em Journal of Biomechanics}, Vol. 43, pp. 1835-1839

\bibitem{schugart}[2004] M. A. Haider and R. C. Schugart, A numerical method for the continuous spectrum biphasic poroviscoelastic model of articular cartilage, {\em Journal of Biomechanics}, Vol. 39, No. 1, pp. 177-183.

\bibitem{gibson}[2005] H. T. Banks, N. L. Gibson, and W. P. Winfree, "Gap detection with electromagnetic terahertz signals", Nonlinear Analysis: Real World Applications 6, no. 2, 381-416.
}
\end{thebibliography}
\end{block}

\begin{block}{Acknowledgements}
This work was funded by a grant from OSU's Undergraduate Research, Innovation, Scholarship \& Creativity (URISC): Start program.
\end{block}
\end{column}
\end{columns}

\end{frame}
\end{document}
