\documentclass[12pt,letterpaper]{article}

\author{Jordan Bayles}
\date{May 7th, 2012} % Necessary?
\title{Numerical solutions implemented in determining values of $v$, $\dot{Q}$}

%Usepackage declarations
\usepackage[left=1in,top=1in,right=1in,bottom=1in]{geometry}
\usepackage[T1]{fontenc}
\usepackage{tgpagella}
\usepackage[protrusion=true,expansion=true]{microtype}
\usepackage[usenames,dvipsnames]{color}
\usepackage{fancyhdr}
\usepackage{lastpage}
\usepackage{sectsty}
\usepackage{slashed}
\usepackage{amsmath}
\usepackage{amsfonts}
\usepackage{latexsym}
% Include for use of images
\usepackage{graphicx}
% Include for use of [H] placement specifier
\usepackage{float}
% Include for use of \toprule, \midrule, \bottomrule in tabular env.
\usepackage{booktabs}
% Include for setting spacing between lines
\usepackage{setspace}
%Package usages
\sectionfont{\normalsize}
\subsectionfont{\small}

%New commands
\newcommand{\comment}[1]{}
\newcommand{\field}[1]{\mathbb{#1}} % requires amsfonts
\newcommand{\pd}[2]{\frac{\partial#1}{\partial#2}}
\newcommand{\script}[1]{\mathcal{#1}} % requires amsfonts, for clarity
\begin{document}
\maketitle

\noindent In order to determine $v$, defined as
% Double check this definition, I also have v = \kappa \pd{\sigma}{z}
\begin{equation}
v = \pd{u}{t}
\end{equation}
for use with an advection/diffusion model, we need to (1) characterize
$\dot{\script{Q}}$, defined as
\begin{equation}
\script{Q} = \pd{\script{P}}{z}
\end{equation}
and then (2) use this characterization to solve for $v$. It was determined
that a numerical solution should be used, specifically using finite difference equations.

\section{Finding $\dot{\script{Q}}$}
$\dot{\script{Q}}$
can be determined using the forward difference approximation for first derivatives:
\begin{equation}
\dot{Q}_{n + \frac{1}{2}} = \frac{Q_{n+1} - Q_n}{\Delta t}
\end{equation}
Where $\script{Q}$ is determined also using a numerical method, specifically
\begin{equation}
\script{Q}_{n+1} = \script{Q}_n + \Delta t \frac{G_d}{\tau} Av
\end{equation}
where the update step is dependent upon $v$, and can be reformulated and written as
\begin{equation}
\script{Q}_{n+1} = \script{Q}_n + \frac{\Delta t \cdot G_d \cdot v_{zz}}{\tau} -
\frac{\Delta t \script{Q}_n}{\tau}
\end{equation}

% Expand here

\section{Finding $v_{zz}$}
Due to the fact that $v$ and $\script{Q}$ are co-dependent (i.e. their values
depend upon each other) they are thus solved simultaneously. The second order,
center, finite difference approximation of $v_{zz}$ (read: $\pd{^2 v}{z^2}$) is
simplified using the fact that $v$ represents a vector of length $M$ (the number
of nodes approximated at). Using a diffusion coefficient $\gamma$, where
\begin{equation}
\gamma = \frac{\Delta t}{ (\Delta z)^2}
\end{equation}
and a tridiagonal matrix $A$ in order to take the center finite difference for the
values in vector $v$, specifically of the form
\begin{equation}
A = \left[ \begin{array}{c c c c c c c }
-2	& 1		& 0		& 0		& 0		& 0		& 0\\
1	& -2	& 1		& 0		& 0		& 0		& 0\\
0	& 1		& -2	& 1		& 0		& 0		& 0\\
0	& 0		&\ddots	&\ddots	&\ddots	& 0		& 0\\
0	& 0		& 0		&\ddots	&\ddots	&\ddots	& 0\\
0	& 0		& 0		& 0		& 1		& -2	& 1\\
0	& 0		& 0		& 0		& 0		& 1		& -2\\
\end{array} \right]
\end{equation}

Then it follows simply that
\begin{equation}
\vec{v}_{zz} = A \vec{v}
\end{equation}

\section{Finding $v$}
After a solution for $v_{zz}$ is determined, it can be used in determining the
value of $v$ (as this is done numerically, this is \emph{iteratively} determined),
specifically as
\begin{equation}
v_n = v_{n-1} + \kappa H_A G_{\infty} \cdot v_{zz} + \Delta t \cdot \kappa H_A 
\dot{\script{Q}}
\end{equation}

\section{Ignore past here}
% We currently are not utilizing this, as j iterates over the empty set.
$v$ is also solved using finite difference, however this is done using a second
order central difference approximation, defined as
\begin{equation}
f'' (x) = \frac{f(x+h) - 2 \cdot f(x) + f(x-h)}{h^2}
\end{equation}
Specifically of the form
\begin{equation}
v_n (j) = v_{n-1}(j) + D \gamma \left(v_{n-1}(j+1)- 2 \cdot v_{n-1}(j) + 
v_{n-1}(j-1) \right) + \Delta t \kappa \dot{\script{Q}}(j)
\end{equation}

Although it may appear that we are missing the $h^2$ term, due to the fact that
$h=1$, this is simply $h^2 = 1$, which is not necessary to include.
\end{document}
