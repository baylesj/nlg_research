\documentclass[12pt,letterpaper]{article}

\author{Jordan Bayles}

%Usepackage declarations
\usepackage[left=1in,top=1in,right=1in,bottom=1in]{geometry}
\usepackage[T1]{fontenc}
\usepackage{tgpagella}
\usepackage[protrusion=true,expansion=true]{microtype}
\usepackage[usenames,dvipsnames]{color}
\usepackage{fancyhdr}
\usepackage{lastpage}
\usepackage{sectsty}
\usepackage{slashed}
\usepackage{amsmath}
\usepackage{amsfonts}
\usepackage{latexsym}
\usepackage{colortbl}
% Include for use of images
\usepackage{graphicx}
% Include for use of [H] placement specifier
\usepackage{float}
% Include for use of \toprule, \midrule, \bottomrule in tabular env.
\usepackage{booktabs}
%Package usages
\sectionfont{\normalsize}
\subsectionfont{\small}

%New commands
\newcommand{\comment}[1]{}
\newcommand{\field}[1]{\mathbb{#1}} % requires amsfonts
\newcommand{\pd}[2]{\frac{\partial#1}{\partial#2}}

\begin{document}
\begin{flushright}
Jordan Bayles\\
Advisor Nathan Gibson
\end{flushright}

\begin{center}
Research Project Overview
\end{center}

\section{Concept overview}
\noindent Given
\[ \sigma (z, t) = H_A \int_0^t G(t-s) \frac{d\epsilon}{ds} (z,s) ds = H_A G * \frac{d\epsilon}{dt} \]
where
\[ G(t) = 1 + c \int_{\tau_1}^{\tau_2} g(t, \tau) dF( \tau ) = G_{\infty} + G_d \field{E}_F [g] \]
In other words, $G_{\infty} \equiv 1, G_d \equiv c$ with
\[ g (t, \tau) = \frac{ e^{-t/\tau}}{\tau}, \quad \tau = \textnormal{ "relaxation time"}\]
Note that
\[ \hat{g} = \frac{1}{ 1 + i\omega\tau} \]
Then, $\sigma(z,t)$ can be expressed as
\[ \sigma(z, t) = H_A G_{\infty} \left. \epsilon \right|_0^t + H_A G_d P \]
where $P := \field{E}_F [g] * d\epsilon/dt$, giving
\[ \hat{P} = \field{E}_F [ \hat{g} ] \cdot \frac{d\epsilon}{dt} = \underbrace{\int_{\tau_1}^{\tau_2} \frac{1}{1 + i\omega\tau} d F(\tau)}_{``\field{E}(\omega)"} \cdot \frac{d\hat{\epsilon}}{dt} \]
``Can show that'' $\mathcal{P}$ satisfies an (auxiliary) ODE:
\[ \tau \dot{\mathcal{P}} + \mathcal{P} = G_d \frac{d\epsilon}{dt},\textnormal{ where } \tau \sim F(\tau) \]
Where $P = \field{E}_F [ \mathcal{P} ]$.

\section{Project overview}
\begin{table}[H]
\begin{tabular}{>{\bfseries}l c}\toprule
Weeks & Planned Activities\\\midrule
1, 2 & Show problems are analogous (auxiliary ODE same, although PDEs differ)\\
3, 4 & Implement polynomial chaos \\
5... & Rigorously prove approach is valid; recover Gauss-Legendre as special case
\\\bottomrule
\end{tabular}
\end{table}
\end{document}