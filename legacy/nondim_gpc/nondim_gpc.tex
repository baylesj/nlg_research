\documentclass[12pt,letterpaper]{article}

\author{}
\date{}
\title{}

%Usepackage declarations
\usepackage[left=1in,top=1in,right=1in,bottom=1in]{geometry}
\usepackage[usenames,dvipsnames]{color}
\usepackage{fancyhdr}
\usepackage{lastpage}
\usepackage{sectsty}
\usepackage{slashed}
\usepackage{amsmath}
\usepackage{amsfonts}
\usepackage{latexsym}
% Include for use of images
\usepackage{graphicx}
% Include for use of [H] placement specifier
\usepackage{float}
% Include for use of \toprule, \midrule, \bottomrule in tabular env.
\usepackage{booktabs}
% Include for setting spacing between lines
\usepackage{setspace}
\usepackage{parskip}
%Package usages
\sectionfont{\normalsize}
\subsectionfont{\small}

%%Fancy Header setup

\pagestyle{fancy}
% Clear default
\fancyhead{}
\fancyfoot{}
%New settings
\fancyhead[R]{Jordan Bayles}
\fancyhead[C]{Nondimensionalized application of gPC}
\fancyfoot[C]{\thepage}
\renewcommand{\headrulewidth}{0.4pt}
\renewcommand{\footrulewidth}{0.4pt}

%New commands
\newcommand{\comment}[1]{}
\newcommand{\field}[1]{\mathbb{#1}} % requires amsfonts
\newcommand{\script}[1]{\mathcal{#1}} % requires amsfonts
\newcommand{\pd}[2]{\frac{\partial#1}{\partial#2}}
\newcommand{\m}[3]{\left[#1_{ij}\right]_{#2\times #3}}
\newcommand{\mt}[4]{\left(\left[#1_{ij}\right]_{#2\times #3}^{#4}\right)^T}
\newcommand{\vn}[3]{\vec{#1}_{#2\times #3}}
\newcommand{\vt}[4]{\left(\vec{#1}_{#2\times #3}^{#4}\right)^T}

\begin{document}
We begin by stating the nondimensionalized form of the PDE model
\begin{equation} \label{eq:ndpdemodel}
\pd{\bar{v}}{\bar{t}} = \left(G_{\infty} + \frac{G_d}{\tau}\right) +
\pd{^2\bar{v}}{\bar{z}^2} - \frac{t_c}{\tau} \pd{^2\bar{P}}{\bar{z}^2}
\end{equation}

With the coupled auxiliary ODE
\begin{equation} \label{eq:auxode}
\tau \pd{\bar{P}}{\bar{t}} = G_d \bar{v} - \frac{1}{\tau} \bar{P}
\end{equation}

Using the following change of variables
\begin{equation*} \label{eq:changeofvariables}
u=h\bar{u},\;z=h\bar{z},\;\sigma = H_a\bar{\sigma},\; t = t_c \bar{t},\;P=\rho\bar{P}
\end{equation*}

And setting $t_c = h^2/(\kappa H_A)$ and $\rho = h$. We then define
$\bar{P}$ as the expected value of $\script{P}$, i.e.
\begin{equation} \label{eq:pexpsp}
\bar{P} = \field{E}_F \left[\bar{P}(\bar{z},\bar{t})\right] :=
G_d \cdot \bar{g} (\bar{t}; \tau) * \bar{v}
\end{equation}

Where $\bar{\script{P}}$ satisfies the auxiliary ODE
\begin{equation} \label{eq:scriptpode}
\tau \pd{\bar{\script{P}}}{\bar{t}} + t_c \bar{\script{P}} = G_d \bar{v},\;
\rm{with}\; \tau \sim F
\end{equation}

Applying generalized Polynomial Chaos, we can say $\tau = r\xi +m$, with
$\xi \in [-1,1]$. Now, assuming non-dimensionalization (i.e. ceasing bar
notation)
\begin{equation} \label{eq:auxodewithgpc}
(rM+mI) \pd{\vec{\alpha}}{t} + t_c\vec{\alpha} = G_d v(z,t) \hat{e}_1,\;
\rm{with}\; \script{P}(t,z,\xi) = \sum_{j=0}^{p} a_j(z,t)\phi_j(xi) 
\end{equation}

Where $\field{E}_F \left[\script{P}\right] = \alpha_0 (t,z)$. Introducing
matrix notation of the form such that an $n\times m$ matrix $A$ is denoted
as $\left[A_{ij}\right]_{m\times n}$, we can write $\alpha$ in the form
\begin{equation}
\alpha_j(t_n,z_i) = \m{\alpha}{m}{p} =
\left[ \begin{array}{cccc}
\alpha_0 (z_0)	&	\alpha_1 (z_0)	&	\cdots	&	\alpha_p (z_0)	\\
\alpha_0 (z_1)	&	\ddots			&			&				  	\\
\vdots			&					&			&					\\
\alpha_0 (z_m)	&	\cdots			&			&	\alpha_p (z_m)
\end{array}\right]
\end{equation}

Using our new notation method and matrix form, equation
\eqref{eq:auxodewithgpc} can be written in the form
\begin{equation} \label{eq:matrixizedauxodewithgpc}
(r\m{M}{p}{p} + m\m{I}{p}{p}) \pd{}{t} \m{\alpha}{m}{p}^T +
t_c \m{\alpha}{m}{p}^T = G_d \left(\vec{v}_{1\times m+2} \cdot
\hat{e}_1\right)
\end{equation}

Discretizing with respect to time, this becomes
\begin{equation} \label{eq:discretizedauxodewithgpc}
(r\m{M}{p}{p} + m\m{I}{p}{p}) \frac{\mt{\alpha}{m}{p}{n+1} - \mt{\alpha}{m}{p}{n}}{\Delta t} + t_c \mt{\alpha}{m}{p}{n+\theta} = G_d \hat{e}_1 
\vt{v}{1}{m+2}{n+\theta}
\end{equation}
\end{document}